\documentclass[slides]{subfiles}

\begin{document}
    \begin{frame}[label=working]{(Discontinuous) Constituency Trees}
        \begin{minipage}{.4\linewidth}
            \resizebox{\linewidth}{!}{
                \subfile{figures/01-ctree-annotations}}
        \end{minipage}
        \begin{minipage}{.58\linewidth}
            \begin{itemize}
                \item<+-> syntactic structure of natural language
                \begin{itemize}
                    \item<+-> lowest level: sentence
                    \item<+-> 2\(^\text{nd}\) lowest: part-of-speech (POS) tags
                    \item<+-> remainder: constituent symbols
                \end{itemize}
                \item<+-> discontinuity = non consecutive regions
                {\scriptsize\centering\begin{tabular}{llc}
                    \toprule
                    corpus & lang. & disc.\\% & disc.\@ nodes \\
                    \midrule
                    CGN \citep{hoekstra2001syntactic} & Nl.      & ?? \\
                    DPTB  \citep{EvaKal11} & Eng.      & $\approx$ 20\% \\
                    LASSY \citep{Noord09} & Nl.      & ?? \\
                    NEGRA \citep{Skut98}   & Ger.      & $\approx$ 27\% \\
                    TIGER \citep{Brants04} & Ger.      & $\approx$ 27\% \\
                    \bottomrule
                \end{tabular}}
                \item<+-> parsing = prediction of structure for sentence
                \item<+-> building block in higher-level NLP tasks, e.g.\@
                \begin{itemize}
                    \item translation \citep{Zhang19, Yang22}
                    \item relation extraction \citep{Nguyen19}
                    \item summarization \citep{Balachandran21}
                \end{itemize}
            \end{itemize}
        \end{minipage}
    \end{frame}
\end{document}