\documentclass[aspectratio=169, 10pt]{beamer}

\usepackage{booktabs, libertinus, subfiles, tikz, colortbl, multirow, natbib}
\usetikzlibrary{positioning, matrix, decorations, fit, decorations.pathreplacing, calc, backgrounds}
\newcommand{\nt}[1]{#1}
\newcommand{\term}[1]{#1}
\newcommand{\x}[1]{#1}
\newcommand{\wildcard}{\ensuremath{*}}
\newcommand{\iparam}{\ensuremath{\alpha}}


\tikzset{
    invisible/.style={opacity=0},
    visible on/.style={alt=#1{}{invisible}},
    defocus on/.style={alt=#1{color=lightgray}{color=black}},
    focus on/.style={alt=#1{color=black}{color=lightgray}},
    alarm on/.style={alt=#1{color=red}{color=black}},
    alt/.code args={<#1>#2#3}{%
        \alt<#1>{\pgfkeysalso{#2}}{\pgfkeysalso{#3}} % \pgfkeysalso doesn't change the path
    },
}

\tikzset{
    cross/.style={
        cross out,
        minimum size=2*(#1-\pgflinewidth),
        inner sep=0pt,
        outer sep=0pt
    },
    cross/.default={1pt},
    triangle/.style={regular polygon, regular polygon sides=3, draw}
}


\tikzset{
    process/.style={draw, rectangle, inner xsep=3pt, inner ysep=0pt},
    boundingbox/.style={inner sep=0pt, outer sep=0pt},
    olabel/.style={gray, inner sep=0pt},
    obox/.style={draw, rectangle, rounded corners, line width=.3mm, align=left}
}


\beamertemplatenavigationsymbolsempty
\defbeamertemplate{footline}{centered frame number}
{%
    \hspace*{\fill}%
    \usebeamercolor[fg]{frame number in head/foot}%
    \usebeamerfont{frame number in head/foot}%
    \insertframenumber\,/\,\inserttotalframenumber%
    \hspace*{\fill}\vskip2pt%
}
\setbeamertemplate{footline}[center frame number]

\title{Supertagging-based Parsing\\for Discontinuous Constituency Trees}
\subtitle{Status talk}
\newcommand{\theemails}{%
    \href{thomas.ruprecht@tu-dresden.de}{thomas.ruprecht@tu-dresden.de}%
}
\author{\texorpdfstring{Thomas Ruprecht\\{\small \theemails}}{Thomas Ruprecht}}
\institute{Institute for Theoretical Computer Science\\Faculty of Computer Science\\Technische Universität Dresden, Germany}
\date{Mai 9, 2023}

\begin{document}
    \maketitle

    \begin{frame}{Outline}
        \tableofcontents
    \end{frame}

    \section{Introdcution}
    \begin{frame}{(Discontinuous) Constituency Trees}
        \begin{minipage}{.4\linewidth}
            \resizebox{\linewidth}{!}{
                \subfile{figures/01-ctree-annotations}}
        \end{minipage}
        \begin{minipage}{.58\linewidth}
            \begin{itemize}
                \item<+-> syntactic structure of natural language
                \begin{itemize}
                    \item<+-> lowest level: sentence
                    \item<+-> 2\(^\text{nd}\) lowest: part-of-speech tags
                    \item<+-> remainder: constituent symbols
                \end{itemize}
                \item<+-> discontinuity = non consecutive regions
                {\scriptsize\centering\begin{tabular}{llcc}
                    \toprule
                    corpus & language & disc.\@ sent.\@ & disc.\@ nodes \\
                    \midrule
                    Alpino & nl       &                 &               \\
                    DPTB   & eng      &                 &               \\
                    NeGra  & ger      &                 &               \\
                    Tiger  & ger      &                 &               \\
                    \bottomrule
                \end{tabular}}
                \item<+-> parsing = prediction of structure for sentence
                \item<+-> improves nat.\@ lang.\@ understanding and generation
            \end{itemize}
        \end{minipage}
    \end{frame}

    \begin{frame}<1-9>{Hybrid Grammars \cite{VogDre,Geb,Geb}}
        \begin{minipage}{.58\linewidth}
            \begin{overprint}
                \onslide<-6>
                    \resizebox{\linewidth}{!}{
                        \subfile{figures/03-lexderiv-overlays}}
                \onslide<7->
                    \resizebox{\linewidth}{!}{
                        \subfile{figures/06-grammar-training-pipeline}}\\
                    \resizebox{\linewidth}{!}{
                        \subfile{figures/04-grammar-pipeline}}
            \end{overprint}
        \end{minipage}
        \begin{minipage}{.4\linewidth}
            \begin{itemize}
                \item rule
                    \begin{itemize}
                        \item nonterminals
                        \item compositions
                            \begin{itemize}
                                \item[(a)] over string tuples
                                \item[(b)] over (consecutive) tree sequences
                            \end{itemize}
                        \item alignment
                    \end{itemize}
                \item parsing:
                \begin{itemize}
                    \item find derivation via (a)
                    \item comp.\@ constituent tree via (b)
                \end{itemize}

                \item constituent tree → rule derivation
                \begin{itemize}
                    \item rule for each inner node
                    \item constituent symbols → nonterminals
                \end{itemize}
                \item treebank → weighted grammar (cost for each rule) \cite{}
                \item but: parsing is tedious (slow and inaccurate by today's standards) \cite{}
                \item complexity dominated by the size of extracted grammars \cite{}
            \end{itemize}
        \end{minipage}

        examples for parsing with grammars vs. sota approaches
        \begin{tabular}{lccc}
            \toprule
            \midrule
            Gebhard\\
            van Cranenbourgh\\
            \midrule
            Corro\\
            Coavoux\\
            FG-GR\\
            \bottomrule
        \end{tabular}
    \end{frame}


    \begin{frame}{Supertagging -- Guessing a Small Grammar}
        -- ablaufdiagramm --
        \begin{itemize}
            \item prediction model -- sentence to supertags
            \item parsing
            \item derivation to constituent tree
            \item offene Fragen:
            \begin{itemize}
                \item woher kommen die blueprints?
                \item woher kommt das prediction model?
                \item wie wird der constituent tree aus der lexikalen derivation abgelesen?
            \end{itemize}
            \item 1 und 3 sind eng miteinander verbunden
            \item
        \end{itemize}
    \end{frame}


    \section{Lexicalized Rule extraction}
    \begin{frame}{Supertag Extraction by LCFRS Rule Induction and Lexicalization}
        -- aus naacl slides --
        \begin{itemize}
            \item behalte pos tags bei fuse
            \item am ende nochmal die Supertags als komponenten (pos, swapped, regel) darstellen
        \end{itemize}
    \end{frame}


    \section{Supertag Prediction and Parsing}
    \begin{frame}{Prediction Model}
        \begin{itemize}
            \item nur kurz was zu Bert mit aufgesetztem Kopf
        \end{itemize}
    \end{frame}


    \section{Experimental Evaluation and Results}
    \begin{frame}{Results}
        -- aus naacl slides --
    \end{frame}


    \section{Publications}
    \begin{frame}{Publications}
        \begin{itemize}
            \item asdf
        \end{itemize}
    \end{frame}


    \section*{References}
     \begin{frame}[fragile,noframenumbering,plain]{References}
        \bibliography{references}
        \bibliographystyle{acl_natbib}
    \end{frame}
\end{document}